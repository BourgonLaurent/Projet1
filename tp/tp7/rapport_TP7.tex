\documentclass[12pt]{scrartcl}

\usepackage[utf8]{inputenc}
\usepackage[letterpaper]{geometry}
\usepackage{fancyhdr}
\usepackage{csquotes}
\usepackage[french]{babel}

\pagestyle{fancy}
% \fancyhead[R]{} - Active if we don't want the base fancy header
\fancyfoot[L]{INF1900}
\fancyfoot[R]{Librairies et débogage}

\author{Laurent Bourgon \\Mehdi Benouhoud \\Ihsane Majdoubi \\Catalina Andrea Araya Figueroa}
\subtitle{Travaux pratiques 7 et 8}
\title{Production de librairie statique et stratégie de débogage}

\date{Lundi 13 mars 2023}


\begin{document}
\maketitle



\newpage

\section{Description de la librairie}

\subsection{Classes}
\subsubsection{LED}
Depuis le début du cours, nous utilisons un diode électro-luminescente située
sur le circuit imprimé du robot. Cette diode est dite  \textquote{bi-colore}, c'est-à-dire
qu'elle peut s'allumer en rouge ou en vert, dépendament du sens du courant qu'on
lui transmets. En faisant clignoter rapidement la diode en alternant la couleur,
on obtient une troisième couleur: ambrée.
% METTRE DIAGRAMME DE CLASSE ICI
\newpage
\section{Modifications apportées au \textit{Makefile} de départ}

\end{document}